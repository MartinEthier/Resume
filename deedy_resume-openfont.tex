%%%%%%%%%%%%%%%%%%%%%%%%%%%%%%%%%%%%%%%
% Deedy - One Page Two Column Resume
% LaTeX Template
% Version 1.2 (16/9/2014)
%
% Original author:
% Debarghya Das (http://debarghyadas.com)
%
% Original repository:
% https://github.com/deedydas/Deedy-Resume
%
% IMPORTANT: THIS TEMPLATE NEEDS TO BE COMPILED WITH XeLaTeX
%
% This template uses several fonts not included with Windows/Linux by
% default. If you get compilation errors saying a font is missing, find the line
% on which the font is used and either change it to a font included with your
% operating system or comment the line out to use the default font.
% 
%%%%%%%%%%%%%%%%%%%%%%%%%%%%%%%%%%%%%%
% 
% TODO:
% 1. Integrate biber/bibtex for article citation under publications.
% 2. Figure out a smoother way for the document to flow onto the next page.
% 3. Add styling information for a "Projects/Hacks" section.
% 4. Add location/address information
% 5. Merge OpenFont and MacFonts as a single sty with options.
% 
%%%%%%%%%%%%%%%%%%%%%%%%%%%%%%%%%%%%%%
%
% CHANGELOG:
% v1.1:
% 1. Fixed several compilation bugs with \renewcommand
% 2. Got Open-source fonts (Windows/Linux support)
% 3. Added Last Updated
% 4. Move Title styling into .sty
% 5. Commented .sty file.
%
%%%%%%%%%%%%%%%%%%%%%%%%%%%%%%%%%%%%%%%
%
% Known Issues:
% 1. Overflows onto second page if any column's contents are more than the
% vertical limit
% 2. Hacky space on the first bullet point on the second column.
%
%%%%%%%%%%%%%%%%%%%%%%%%%%%%%%%%%%%%%%


\documentclass[]{deedy-resume-openfont}
\usepackage{fancyhdr}
 
\pagestyle{fancy}
\fancyhf{}
 
\begin{document}

%%%%%%%%%%%%%%%%%%%%%%%%%%%%%%%%%%%%%%
%
%     TITLE NAME
%
%%%%%%%%%%%%%%%%%%%%%%%%%%%%%%%%%%%%%%

\namesection{Martin}{Ethier}{
\href{mailto:methier@edu.uwaterloo.ca}{methier@edu.uwaterloo.ca} |
(705) 303-8890 |
\urlstyle{same}\href{https://www.linkedin.com/in/ethiermartin/}{linkedin.com/in/ethiermartin} |
\urlstyle{same}\href{https://github.com/MartinEthier}{github.com/MartinEthier}
}

%%%%%%%%%%%%%%%%%%%%%%%%%%%%%%%%%%%%%%
%
%     COLUMN ONE
%
%%%%%%%%%%%%%%%%%%%%%%%%%%%%%%%%%%%%%%

\begin{minipage}[t]{0.33\textwidth} 

%%%%%%%%%%%%%%%%%%%%%%%%%%%%%%%%%%%%%%
%     SKILLS
%%%%%%%%%%%%%%%%%%%%%%%%%%%%%%%%%%%%%%

\section{Skills}
\location{Languages}
Python \textbullet{} C++ \textbullet{} MATLAB \textbullet{} Java \\
\location{Libraries and Frameworks}
Scikit-Learn \textbullet{} Tensorflow \textbullet{} Keras \textbullet{} OpenCV \textbullet{} wxPython \textbullet{} Pandas \textbullet{} ROS \\
\location{Hardware}
Raspberry Pi 3 \textbullet{} Arduino

\sectionsep

%%%%%%%%%%%%%%%%%%%%%%%%%%%%%%%%%%%%%%
%     COURSEWORK
%%%%%%%%%%%%%%%%%%%%%%%%%%%%%%%%%%%%%%

\section{Coursework}
\subsection{Undergraduate}
\vspace{\topsep}
\begin{tightemize}
\item Data Structures and Algorithms in C++
\item Microprocessors and Digital Logic
\item Probability and Statistics
\item Ordinary Differential Equations
\item Computational Methods using MATLAB
\end{tightemize}
\sectionsep

\subsection{Online}
\location{\urlstyle{same}\href{https://www.udacity.com/course/artificial-intelligence-for-robotics--cs373}{Artificial Intelligence for Robotics, Udacity}}
%\vspace{\topsep}
\begin{tightemize}
\item Course teaches the algorithms behind self-driving cars using Python
\item Topics covered include graph SLAM, PID control, Particle and Kalman Filters, and A* Searching
\end{tightemize}

\location{\urlstyle{same}\href{https://developers.google.com/machine-learning/crash-course/}{Machine Learning Crash Course, Google}}
\begin{tightemize}
\item Interactive introductory machine learning course taught using Python and Tensorflow
\item Topics covered include feature engineering, regression and classification, neural networks, embeddings, and regularization
\end{tightemize}

\sectionsep

%%%%%%%%%%%%%%%%%%%%%%%%%%%%%%%%%%%%%%
%     EDUCATION
%%%%%%%%%%%%%%%%%%%%%%%%%%%%%%%%%%%%%%

\section{Education} 

\subsection{University of Waterloo}
\descript{Candidate for B.ASc in Mechatronics Engineering}
\location{Sep. 2016 - Apr. 2022}
\sectionsep


%%%%%%%%%%%%%%%%%%%%%%%%%%%%%%%%%%
% Interests
%%%%%%%%%%%%%%%%%%%%%%%%%%%%%%%%%

\section{Interests}
Autonomous Vehicles \textbullet{} Deep Learning \\ \textbullet{} Machine Vision \textbullet{} Artificial Intelligence \\ \textbullet{} Hockey \textbullet{} Weightlifting \textbullet{} Piano

%%%%%%%%%%%%%%%%%%%%%%%%%%%%%%%%%%%%%%
%
%     COLUMN TWO
%
%%%%%%%%%%%%%%%%%%%%%%%%%%%%%%%%%%%%%%

\end{minipage} 
\hfill
\begin{minipage}[t]{0.66\textwidth} 

%%%%%%%%%%%%%%%%%%%%%%%%%%%%%%%%%%%%%%
%     EXPERIENCE
%%%%%%%%%%%%%%%%%%%%%%%%%%%%%%%%%%%%%%

\section{Experience}

\runsubsection{Information Processing Co-op}\\
\descript{\urlstyle{same}\href{http://www.crc.gc.ca/eic/site/069.nsf/eng/home}{Communications Research Centre}}
\location{Sep. 2018 – Dec. 2018 | Kanata, ON}
\vspace{\topsep} % Hacky fix for awkward extra vertical space
\begin{tightemize}
\item Developed location data visualization tools using wxPython and the Google Maps Static API, allowing researchers for quick analysis of the data
\item Used Keras and Scikit-Learn to develop new spatial interpolation algorithms for location data with neural networks and random forest ensemble methods 
\item Introduced rotational features in data to reduce error and smooth out interpolated map

\end{tightemize}
\sectionsep

\runsubsection{CT Analysis Research Assistant}\\
\descript{\urlstyle{same}\href{http://msam-uwaterloo.ca/}{Multi-Scale Additive Manufacturing Lab}}
\location{Jan. 2018 – Apr. 2018 | Waterloo, ON}
\begin{tightemize}
\item Worked with the Matlab Image Processing Toolbox to develop software for analysis of CT scanned image stacks of 3D printed metal parts
\item Applied techniques such as histogram thresholding, flood fill, and watershed segmentation in order to acquire porosity data from the parts
\item Imaged the parts using a Nano-CT Scanner and a Laser Scanning Confocal Microscope
\end{tightemize}
\sectionsep

\runsubsection{Medical Physics Assistant}\\
\descript{\urlstyle{same}\href{https://www.lhsc.on.ca/}{London Health Sciences Centre}}
\location{May 2018 - Aug. 2018 | London, ON}
\begin{tightemize}
\item Irradiated radio-chromic gels using medical linear accelerators and analyzed images of the gels in Matlab to get dose measurements 
\item Created IronPython scripts to automate and streamline the creation of radiation treatment plans
\item Used WPF and XAML to incorporate custom GUIs into the software
\end{tightemize}
\sectionsep

%%%%%%%%%%%%%%%%%%%%%%%%%%%%%%%%%%%%%%
%     TEAMS
%%%%%%%%%%%%%%%%%%%%%%%%%%%%%%%%%%%%%%

\section{Activities}
\runsubsection{\urlstyle{same}\href{https://uwrobotics.uwaterloo.ca/}{UW Robotics}}
\descript{| Software Team Member}
\location{Jan. 2019 – Present | Waterloo, ON}
\begin{tightemize}
\item Part of the student team competing in the International Autonomous Robot Racing Competition
\item Developing self-driving robotics C++ software architecture in a Linux environment using the ROS framework along with OpenCV
\item Utilizing CUDA to integrate the YOLOv3 single shot detection CNN model in order to detect and classify navigation signs on the race track
\end{tightemize}
\sectionsep

%%%%%%%%%%%%%%%%%%%%%%%%%%%%%%%%%%%%%%
%     PROJECTS
%%%%%%%%%%%%%%%%%%%%%%%%%%%%%%%%%%%%%%

\section{Projects}
\runsubsection{\urlstyle{same}\href{https://github.com/MartinEthier/DogBreedClassifier}{Dog Breed Classifier}} \\
\location{Sep. 2018 – Oct. 2018}
\begin{tightemize}
\item Built an image classifier that returns the most likely dog breeds for the picture
\item Utilized transfer learning in Keras to implement Google's InceptionV3 CNN model using data set of over 20000 images
\item Implemented a data augmentation pipeline to help increase test accuracy to over 80 percent 
\end{tightemize}
\sectionsep


\end{minipage} 
\end{document}  \documentclass[]{article}
