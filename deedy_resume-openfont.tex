%%%%%%%%%%%%%%%%%%%%%%%%%%%%%%%%%%%%%%%
% Deedy - One Page Two Column Resume
% LaTeX Template
% Version 1.2 (16/9/2014)
%
% Original author:
% Debarghya Das (http://debarghyadas.com)
%
% Original repository:
% https://github.com/deedydas/Deedy-Resume
%
% IMPORTANT: THIS TEMPLATE NEEDS TO BE COMPILED WITH XeLaTeX
%
% This template uses several fonts not included with Windows/Linux by
% default. If you get compilation errors saying a font is missing, find the line
% on which the font is used and either change it to a font included with your
% operating system or comment the line out to use the default font.
% 
%%%%%%%%%%%%%%%%%%%%%%%%%%%%%%%%%%%%%%
% 
% TODO:
% 1. Integrate biber/bibtex for article citation under publications.
% 2. Figure out a smoother way for the document to flow onto the next page.
% 3. Add styling information for a "Projects/Hacks" section.
% 4. Add location/address information
% 5. Merge OpenFont and MacFonts as a single sty with options.
% 
%%%%%%%%%%%%%%%%%%%%%%%%%%%%%%%%%%%%%%
%
% CHANGELOG:
% v1.1:
% 1. Fixed several compilation bugs with \renewcommand
% 2. Got Open-source fonts (Windows/Linux support)
% 3. Added Last Updated
% 4. Move Title styling into .sty
% 5. Commented .sty file.
%
%%%%%%%%%%%%%%%%%%%%%%%%%%%%%%%%%%%%%%%
%
% Known Issues:
% 1. Overflows onto second page if any column's contents are more than the
% vertical limit
% 2. Hacky space on the first bullet point on the second column.
%
%%%%%%%%%%%%%%%%%%%%%%%%%%%%%%%%%%%%%%


\documentclass[]{deedy-resume-openfont}
\usepackage{fancyhdr}
 
\pagestyle{fancy}
\fancyhf{}
 
\begin{document}

%%%%%%%%%%%%%%%%%%%%%%%%%%%%%%%%%%%%%%
%
%     TITLE NAME
%
%%%%%%%%%%%%%%%%%%%%%%%%%%%%%%%%%%%%%%

\namesection{Martin}{Ethier}{
\href{mailto:methier@edu.uwaterloo.ca}{methier@edu.uwaterloo.ca} |
(705) 303-8890 |
\urlstyle{same}\href{https://www.linkedin.com/in/ethiermartin/}{linkedin.com/in/ethiermartin} |
\urlstyle{same}\href{https://github.com/MartinEthier}{github.com/MartinEthier}
}

%%%%%%%%%%%%%%%%%%%%%%%%%%%%%%%%%%%%%%
%
%     COLUMN ONE
%
%%%%%%%%%%%%%%%%%%%%%%%%%%%%%%%%%%%%%%

\begin{minipage}[t]{0.33\textwidth} 

%%%%%%%%%%%%%%%%%%%%%%%%%%%%%%%%%%%%%%
%     SKILLS
%%%%%%%%%%%%%%%%%%%%%%%%%%%%%%%%%%%%%%

\section{Skills}
\location{Languages}
Python \textbullet{} C++ \textbullet{} MATLAB \textbullet{} Java
\textbullet{} Visual Basic \textbullet{} XAML \textbullet{} \LaTeX\ \\
\location{Libraries and Frameworks}
Scikit-Learn \textbullet{} Tensorflow \textbullet{} Keras \textbullet{} OpenCV \textbullet{} wxPython \textbullet{} Pandas \textbullet{} ROS \\
\location{Hardware}
Raspberry Pi 3 \textbullet{} Arduino \textbullet{} NVIDIA Jetson

\sectionsep

%%%%%%%%%%%%%%%%%%%%%%%%%%%%%%%%%%%%%%
%     COURSEWORK
%%%%%%%%%%%%%%%%%%%%%%%%%%%%%%%%%%%%%%

\section{Coursework}
\subsection{Undergraduate}
Data Structures and Algorithms \\
Microprocessors and Digital Logic \\
Probability and Statistics \\
Ordinary Differential Equations \\
Computational Methods \\
Circuits \\
\sectionsep

\subsection{Online}
\location{Artificial Intelligence for Robotics, Udacity}
\vspace{\topsep}
\begin{tightemize}
\item Course teaches the algorithms behind self-driving cars using Python
\item Topics covered include graph SLAM, PID control, Particle and Kalman Filters, and A* Searching
\item Wrote software to control a robot with the goal of autonomously chasing another robot in a simulated environment for final project
\end{tightemize}

\location{Machine Learning Crash Course, Google}
\begin{tightemize}
\item Interactive introductory machine learning course taught using Python and Tensorflow
\item Topics covered include feature engineering, regression and classification, neural networks, embeddings, and regularization
\end{tightemize}

\sectionsep

%%%%%%%%%%%%%%%%%%%%%%%%%%%%%%%%%%%%%%
%     EDUCATION
%%%%%%%%%%%%%%%%%%%%%%%%%%%%%%%%%%%%%%

\section{Education} 

\subsection{University of Waterloo}
\descript{Candidate for B.ASc in Mechatronics Engineering}
\location{Sep. 2016 -  | Waterloo, ON}
\sectionsep


%%%%%%%%%%%%%%%%%%%%%%%%%%%%%%%%%%
% Interests
%%%%%%%%%%%%%%%%%%%%%%%%%%%%%%%%%

\section{Interests}
Autonomous Vehicles \textbullet{} Deep Learning \\ \textbullet{} Machine Vision \textbullet{} Artificial Intelligence \\ \textbullet{} Hockey \textbullet{} Weightlifting \textbullet{} Piano

%%%%%%%%%%%%%%%%%%%%%%%%%%%%%%%%%%%%%%
%
%     COLUMN TWO
%
%%%%%%%%%%%%%%%%%%%%%%%%%%%%%%%%%%%%%%

\end{minipage} 
\hfill
\begin{minipage}[t]{0.66\textwidth} 

%%%%%%%%%%%%%%%%%%%%%%%%%%%%%%%%%%%%%%
%     EXPERIENCE
%%%%%%%%%%%%%%%%%%%%%%%%%%%%%%%%%%%%%%

\section{Experience}

\runsubsection{Information Processing Coop}\\
\descript{Communications Research Centre}
\location{Sep. 2018 – Dec. 2018 | Kanata, ON}
\vspace{\topsep} % Hacky fix for awkward extra vertical space
\begin{tightemize}
\item Developed cell network map data visualization tools using wxPython and the Google Maps Static API, allowing researchers for quick analysis of the data
\item Used Keras and Scikit-Learn to develop new spatial interpolation algorithms for cell network measurement data with neural networks and random forest 
\item Introduced rotational features in data to reduce error and smooth out interpolated map

\end{tightemize}
\sectionsep

\runsubsection{CT Analysis Research Assistant}\\
\descript{Multi-Scale Additive Manufacturing Lab}
\location{Jan. 2018 – Apr. 2018 | Waterloo, ON}
\begin{tightemize}
\item Utilized the Matlab Image Processing Toolbox to develop complex image processing software for analysis of CT scanned image stacks of 3D printed metal parts
\item Used techniques such as histogram thresholding, flood fill, and watershed segmentation in order to acquire 3D porosity data from the parts
\item Imaged the parts using a Nano-CT Scanner and a Laser Scanning Confocal Microscope
\end{tightemize}
\sectionsep

\runsubsection{Medical Physics Assistant}\\
\descript{London Health Sciences Centre} 
\location{May 2018 - Aug. 2018 | London, ON}
\begin{tightemize}
\item Used IronPython to develop scripts designed to automate and streamline various processes involved with the creation of radiation treatment plans 
\item Utilized Visual Studio with XAML to design custom GUIs that were implemented in the scripts
\item Irradiated radio-chromic gels using medical linear accelerators and analyzed images of the gels in Matlab to get near-surface dose measurements   
\end{tightemize}
\sectionsep

%%%%%%%%%%%%%%%%%%%%%%%%%%%%%%%%%%%%%%
%     TEAMS
%%%%%%%%%%%%%%%%%%%%%%%%%%%%%%%%%%%%%%

\section{Teams}
\runsubsection{UW Robotics}
\descript{| Software Team Member}
\location{Jan. 2019 – Present | Waterloo, ON}
\begin{tightemize}
\item ROS, C++, Python, Jetson, OpenCV
\end{tightemize}
\sectionsep

%%%%%%%%%%%%%%%%%%%%%%%%%%%%%%%%%%%%%%
%     PROJECTS
%%%%%%%%%%%%%%%%%%%%%%%%%%%%%%%%%%%%%%

\section{Projects}
\runsubsection{Dog Breed Classifier}
\descript{| Android App}
\location{Sep. 2018 – Present}
\begin{tightemize}
\item App allows user to take a picture of their dog and tells them what breed of dog it is, including possible mixes
\item Utilized Keras transfer learning to implement Google's InceptionV3 CNN model using data set of ~20000 labeled images
\item Implemented data augmentation to get an accuracy of over 80 percent on testing data
\item Currently designing Android app using Android Studio and Java to integrate model
\end{tightemize}
\sectionsep


\end{minipage} 
\end{document}  \documentclass[]{article}
